\begin{flushright}
HSF-CWP-2017-14
\end{flushright}

\Large
\begin{center}
{\bf Software trigger and event reconstruction}
\end{center}
\vskip 1cm

\normalsize

\hangindent=1cm
{\bf Editors}: Vladimir Vava Gligorov, LPNHE, Universite Pierre et Marie Curie, Universite Paris Diderot, CNRS/IN2P3 (vgligoro@lpnhe.in2p3.fr) and David Lange, Princeton University (David.Lange@cern.ch) 

\vskip 0.2cm
\hangindent=1cm
{\bf Contributors} [incomplete, please add] : 
Johannes Albrecht (TU Dortmund), 
Ken Bloom (University of Nebraska), 
Tommaso Boccali (INFN Sezione di Pisa a, Università di Pisa b, Scuola Normale Superiore di Pisa), 
Antonio Boveia (Ohio State University), 
Michel De Cian (University of Heidelberg), 
Caterina Doglioni (Lund University), 
Agnieszka Dziurda (CERN), 
Markus Elsing (CERN), 
Amir Farbin (University of Texax, Arlington), 
Conor Fitzpatrick (EPFL), 
Frank Gaede (DESY), 
Simon George (Royal Holloway), 
Hadrien Grasland (Universite de Paris-Sud 11), 
Lucia Grillo (University of Manchester), 
Benedikt Hegner (CERN), 
William Kalderon (Lund University), 
Sami Kama (INST), 
Thorsten Kollegger (GSI), 
Patrick Koppenburg (Nikhef), 
Slava Krutelyov (University of California, San Diego), 
Rob Kutschke (Fermilab), 
Walter Lampl (University of Arizona), 
Ed Moyse (University of Massachusetts, Amherst), 
Andrew Norman (Fermilab), 
Marko Petric (CERN), 
Francesco Polci (LPNHE), 
Karolos Potamianos (DESY), 
Gerhard Raven (VU University Amsterdam, Nikhef), 
Fedor Ratnikov (HSE, YSDA), 
Martin Ritter (Max-Planck-Institut für Physik),  
Andrea Rizzi (INFN Sezione di Pisa a, Università di Pisa b, Scuola Normale Superiore di Pisa), 
Eduardo Rodrigues (University of Cincinnati)
David Rousseau (LAL-Orsay), 
Andy Salzburger (CERN), 
Liz Sexton Kennedy(Fermilab), 
Michael D Sokoloff (University of Cincinnati), 
Graeme Stewart (CERN), 
Mohammad Al-Turany (GSI),  
Andrey Ustyuzhanin (HSE, YSDA), 
Brett Viren (Fermilab), 
Mike Williams (MIT), 
Frank Winklmeier (University of Oregon), 
Frank Wuerthwein (University of California, San Diego), 

\vskip 1cm


\hangindent=1cm
{\bf Abstract:} Realizing the physics programs of the planned and/or upgraded high-energy physics (HEP) experiments over the next 10 years will require the HEP community to address a number of challenges in the area of software and computing. For this reason, the HEP software community has engaged in a planning process over the past two years, with the objective of identifying and prioritizing the research and development required to enable the next generation of HEP detectors to fulfill their full physics potential. The aim is to produce a Community White Paper (CWP)~\cite{HSF2017} which will describe the community strategy and a roadmap for software and computing research and development in HEP for the 2020s. The topics of event reconstruction and software triggers were considered by a joint working group and are summarized together in this document.

